%! TEX root = Distributed_Algorithm.tex
\documentclass[letterpaper, 10 pt, conference]{ieeeconf}
\usepackage{blindtext, graphicx}
\usepackage{cite}
\usepackage[cmex10]{amsmath}
\usepackage{graphicx}
\usepackage[utf8]{inputenc}
\pdfminorversion=4
\usepackage{pifont}
\usepackage{marvosym}
\usepackage{tabu}
\usepackage{multirow}
\usepackage{multicol}
\usepackage{amsfonts}
\usepackage{amssymb}
\usepackage{tabulary}
\usepackage{array}
\usepackage{verbatim}
\usepackage[table]{xcolor}
\definecolor{tableShade}{HTML}{DDDDDD}
\usepackage{tabularx}
\usepackage{tikz}
\usepackage{setspace}
\usepackage{comment}
\usepackage{epsfig}
\usepackage{times}
\usepackage{setspace}
\usepackage{floatflt}
\usepackage{epstopdf}
\usepackage{wrapfig}
\usepackage{ifthen}
%\usepackage{amsthm}
\usepackage[font=small,skip=0pt]{caption}

\newtheorem{theorem}{Theorem}[section]
\newtheorem{corollary}{Corollary}[theorem]
\newtheorem{lemma}[theorem]{Lemma}

% For pseudocode writing
\usepackage{algorithm}
\usepackage{algorithmicx}
\usepackage{algpseudocode}
\usepackage{eqparbox}
\algnewcommand{\IfThenElse}[3]{% \IfThenElse{<if>}{<then>}{<else>}
  \State \algorithmicif\ #1\ \algorithmicthen\ #2\ \algorithmicelse\ #3}
\algnewcommand\algorithmicinput{\textbf{Initialize:}}
\algnewcommand\Initialize{\item[\algorithmicinput]}
% \ifCLASSINFOpdf

% \else
 
% \fi

\newcommand{\psdleq}{\preccurlyeq}
\newcommand{\psdgeq}{\succcurlyeq}


\IEEEoverridecommandlockouts                              % This command is only needed if 
                                                          % you want to use the \thanks command

\overrideIEEEmargins                                      % Needed to meet printer requirements.

% See the \addtolength command later in the file to balance the column lengths
% on the last page of the document

% The following packages can be found on http:\\www.ctan.org
%\usepackage{graphics} % for pdf, bitmapped graphics files
%\usepackage{epsfig} % for postscript graphics files
%\usepackage{mathptmx} % assumes new font selection scheme installed
%\usepackage{times} % assumes new font selection scheme installed
%\usepackage{amsmath} % assumes amsmath package installed
%\usepackage{amssymb}  % assumes amsmath package installed

\title{\LARGE \bf
Title
}
% \author{Albert Author$^{1}$ and Bernard D. Researcher$^{2}$% <-this % stops a space
% \thanks{*This work was not supported by any organization}% <-this % stops a space
% \thanks{$^{1}$Albert Author is with Faculty of Electrical Engineering, Mathematics and Computer Science,
%         University of Twente, 7500 AE Enschede, The Netherlands
%         {\tt\small albert.author@papercept.net}}%
% \thanks{$^{2}$Bernard D. Researcheris with the Department of Electrical Engineering, Wright State University,
%         Dayton, OH 45435, USA
%         {\tt\small b.d.researcher@ieee.org}}%
% }

\hyphenation{op-tical net-works semi-conduc-tor}

\newcommand{\squeezeup}{\vspace{-2.5mm}}


\begin{document}
%\title{Weighted Kalman Filter for Attack Detection and Robust State Estimation}

%\author{\IEEEauthorblockN{Michael Shell}
%\IEEEauthorblockA{School of Electrical and\\Computer Engineering\\
%Georgia Institute of Technology\\
%Atlanta, Georgia 30332--0250\\
%Email: http://www.michaelshell.org/contact.html}
%\and
%\IEEEauthorblockN{Homer Simpson}
%\IEEEauthorblockA{Twentieth Century Fox\\
%Springfield, USA\\
%Email: homer@thesimpsons.com}
%\and
%\IEEEauthorblockN{James Kirk\\ and Montgomery Scott}
%\IEEEauthorblockA{Starfleet Academy\\
%San Francisco, California 96678-2391\\
%Telephone: (800) 555--1212\\
%Fax: (888) 555--1212}}

\maketitle
\thispagestyle{empty}
\pagestyle{empty}


%%%%%%%%%%%%%%%%%%%%%%%%%%%%%%%%%%%%%%%%%%%%%%%%%%%%%%%%%%%%%%%%%%%%%%%%%%%%%%%%
% \begin{abstract}

% \end{abstract}



% \section{Introduction} \label{sec:intro}
% \input{secs/Introduction}

\section{Prior Art}
\label{sec:prior}
% Why they do it?  How they do? What they found? What it means? % what are the main differences? 
Dadras et al. \cite{DGRS15} study behavior of a vehicle platoon in an adversarial environment. They consider insider attack where an adversary has control of a vehicle of the platoon and whose derivative controller gain it can modify to destabilize or take control of the entire platoon. In the considered platoon, predecessor and follower vehicles distance and velocity information are used by a vehicle to compute its velocity and acceleration. The attack considered in the paper is feasible at every position of the platoon. 
%\begin{itemize}
% \item Questions and observations made in the paper
% \item Under adversarial environment stability and string stability are not same
% \item Controllability of platoon by a vehicle at different position and if platoon is controllable by the vehicle then 
%\end{itemize}


\section{Preliminaries and Problem Description} 
\label{sec:prelim}
\subsection{Car-Following Model of Truck Platoon}


\subsection{Attack Model}

We consider an adversary that can control $m$ out of $n$ trucks (where, $m \leq n/2$) of a Driver Assistive Truck Platoon (DATP). The trucks uses radar and vehicle-to-vehicle (V2V) communication to drive longitudinally in a close-headway formation at highway speeds and to maintain a safe gap with the preceding vehicle. By remotely exploiting vulnerability in $m$ truck's communication software, an attacker can hijack V2V's message transmission unit. Subsequently, by injecting malicious messages in the communication network of $(n-m)$ trucks, an adversary intends to impact the string stability of the platoon.  We assume that the leader of the platoon is not attacked and we do not know the $m$ attacked vehicles. Thus, the first goal is to detect the attacked trucks. The next goal is to identify the minimum number of trucks $m$, an adversary should attack to destabilize the platoon. 

In the second attack model, we consider multiple vehicles communicating via V2V in a multilane highway. The vehicles broadcast Basic Safety Messages (BSM) to its neighboring vehicles. An adversary in control of one of the V2V equipped car can broadcast corrupted BSM with the intention of influencing the direction of motion of the vehicles. Thus, the goal is to design control law that can guarantee resilient coordinated motion of vehicles \cite{SSPPK17}.       

%\subsection{Problem Description}





% \begin{itemize}
%     \item How does authentication work in VPKI? 
%     \item If a malicious car injects messages, how are the data validated or is any message accepted? 
%     \item Attack on sensor. Attack on network
%     \item How many vehicles are required for consensus algorithm to work correctly? 
%     \item Ramp condition for trucks not considered
% \end{itemize}



% \section{Methodology} \label{sec:method}
% \input{secs/Methodology}

%\section{Experimental Results} \label{sec:demo}
%\input{secs/Results}

%\vspace{-0.1cm}

%\section{Conclusion} \label{sec:conclude}
%\input{secs/6-conclusion}



%\appendices
%\section{Proof of the First Zonklar Equation}


% \section*{Acknowledgment}

% The work presented in this paper is supported by National Science Foundation, grant \#
% \ifCLASSOPTIONcaptionsoff
%   \newpage
% \fi

\bibliographystyle{ieeetr}
\bibliography{cps}


% \section{Notes}\label{sec:notes}
% \input{secs/7-notes}

%\section{Summary of Other Papers}
%\input{secs/2-priorart}
\end{document}

% Input: Observation y' in Algorithm 1 changed from y


