%! TEX root = Distributed_Algorithm.tex
\documentclass[letterpaper, 10 pt, conference]{ieeeconf}
\usepackage{blindtext, graphicx}
\usepackage{cite}
\usepackage[cmex10]{amsmath}
\usepackage{graphicx}
\usepackage[utf8]{inputenc}
\pdfminorversion=4
\usepackage{pifont}
\usepackage{marvosym}
\usepackage{tabu}
\usepackage{multirow}
\usepackage{multicol}
\usepackage{amsfonts}
\usepackage{amssymb}
\usepackage{tabulary}
\usepackage{array}
\usepackage{verbatim}
\usepackage[table]{xcolor}
\definecolor{tableShade}{HTML}{DDDDDD}
\usepackage{tabularx}
\usepackage{tikz}
\usepackage{setspace}
\usepackage{comment}
\usepackage{epsfig}
\usepackage{times}
\usepackage{setspace}
\usepackage{floatflt}
\usepackage{epstopdf}
\usepackage{wrapfig}
\usepackage{ifthen}
%\usepackage{amsthm}
\usepackage[font=small,skip=0pt]{caption}

\newtheorem{theorem}{Theorem}[section]
\newtheorem{corollary}{Corollary}[theorem]
\newtheorem{lemma}[theorem]{Lemma}

% For pseudocode writing
\usepackage{algorithm}
\usepackage{algorithmicx}
\usepackage{algpseudocode}
\usepackage{eqparbox}
\algnewcommand{\IfThenElse}[3]{% \IfThenElse{<if>}{<then>}{<else>}
  \State \algorithmicif\ #1\ \algorithmicthen\ #2\ \algorithmicelse\ #3}
\algnewcommand\algorithmicinput{\textbf{Initialize:}}
\algnewcommand\Initialize{\item[\algorithmicinput]}
% \ifCLASSINFOpdf

% \else
 
% \fi

\newcommand{\psdleq}{\preccurlyeq}
\newcommand{\psdgeq}{\succcurlyeq}


\IEEEoverridecommandlockouts                              % This command is only needed if 
                                                          % you want to use the \thanks command

\overrideIEEEmargins                                      % Needed to meet printer requirements.

% See the \addtolength command later in the file to balance the column lengths
% on the last page of the document

% The following packages can be found on http:\\www.ctan.org
%\usepackage{graphics} % for pdf, bitmapped graphics files
%\usepackage{epsfig} % for postscript graphics files
%\usepackage{mathptmx} % assumes new font selection scheme installed
%\usepackage{times} % assumes new font selection scheme installed
%\usepackage{amsmath} % assumes amsmath package installed
%\usepackage{amssymb}  % assumes amsmath package installed

\title{\LARGE \bf
Title
}
% \author{Albert Author$^{1}$ and Bernard D. Researcher$^{2}$% <-this % stops a space
% \thanks{*This work was not supported by any organization}% <-this % stops a space
% \thanks{$^{1}$Albert Author is with Faculty of Electrical Engineering, Mathematics and Computer Science,
%         University of Twente, 7500 AE Enschede, The Netherlands
%         {\tt\small albert.author@papercept.net}}%
% \thanks{$^{2}$Bernard D. Researcheris with the Department of Electrical Engineering, Wright State University,
%         Dayton, OH 45435, USA
%         {\tt\small b.d.researcher@ieee.org}}%
% }

\hyphenation{op-tical net-works semi-conduc-tor}

\newcommand{\squeezeup}{\vspace{-2.5mm}}


\begin{document}
%\title{Weighted Kalman Filter for Attack Detection and Robust State Estimation}

%\author{\IEEEauthorblockN{Michael Shell}
%\IEEEauthorblockA{School of Electrical and\\Computer Engineering\\
%Georgia Institute of Technology\\
%Atlanta, Georgia 30332--0250\\
%Email: http://www.michaelshell.org/contact.html}
%\and
%\IEEEauthorblockN{Homer Simpson}
%\IEEEauthorblockA{Twentieth Century Fox\\
%Springfield, USA\\
%Email: homer@thesimpsons.com}
%\and
%\IEEEauthorblockN{James Kirk\\ and Montgomery Scott}
%\IEEEauthorblockA{Starfleet Academy\\
%San Francisco, California 96678-2391\\
%Telephone: (800) 555--1212\\
%Fax: (888) 555--1212}}

\maketitle
\thispagestyle{empty}
\pagestyle{empty}


%%%%%%%%%%%%%%%%%%%%%%%%%%%%%%%%%%%%%%%%%%%%%%%%%%%%%%%%%%%%%%%%%%%%%%%%%%%%%%%%
% \begin{abstract}

% \end{abstract}



% \section{Introduction} \label{sec:intro}
% \input{secs/Introduction}

\section{Prior Art}
\label{sec:prior}
% Why they do it?  How they do? What they found? What it means? % what are the main differences? 
Dadras et al. \cite{DGRS15} study behavior of a vehicle platoon in an adversarial environment. They consider insider attack where an adversary has control of a vehicle of the platoon and whose derivative controller gain it can modify to destabilize or take control of the entire platoon. In the considered platoon, predecessor and follower vehicles distance and velocity information are used by a vehicle to compute its velocity and acceleration. The attack considered in the paper is feasible at every position of the platoon. 
%\begin{itemize}
% \item Questions and observations made in the paper
% \item Under adversarial environment stability and string stability are not same
% \item Controllability of platoon by an attacker vehicle at different position: a platoon is controllable by the attacker controlled vehicle only if it has full ranked controllability matrix and the uncontrollable states can be computed by performing similarity comparison.  
%\item The derivative gain of the controller should be within a range of values to cause instability of string
%\item Find the necessary and maximum value of attacker gain to cause instability of platoon ?
%\item Find the control input fequency that will destabilize the platoon?
%\item The platoon can be made asymptotically and BIBO unstable by making real part of single eigenvalue of matrix A > 0. 
%\item The platoon can be made to collapse at one target vehicle by attacking multiple vehicle of the platoon
%\end{itemize}

Sajjad et al. \cite{SSDSG15} have designed an adversarial aware sliding mode control scheme that uses only local sensor data and a decentralized attack detector to reduce the number and severity of collision from an attacker controlled vehicle of a platoon. Their solution makes the assumption that the bidirectional platoon is homogeneous with all cars sharing thesame control law. They consider a single attacker that modifies the control law of a vehicle to induce an oscillatory behavior in the platoon. While preserving safety of the platoon, the method led to compromise of string stability.
%  They start the attack when platoon is in steady state. Size of the platoon is known and the controller gains are tuned according to it
% IN CACC absolute relative distance is measured by radar and acceleration information of the preceding vehicle is received through the wireless channel 

Fanid et al. \cite{FDZZ17} study the impact of channel and mobile jamming attack on the string stability of interconnected vehicles (homogeneous) equipped with  cooperative adaptive cruise control (CACC) system. The attacker (a drone) in their analysis jams the wireless channel among the vehicles to prevent the receivers from decoding the messages and thus cause destabilization of the platoon. They consider a two-ray path loss model to demonstrate that signal's power attenuation during the attack degrades performance of the CACC system and destabilizes the platoon. They found that by increasing headway time and using a memory block in vehicles CACC system, stability could be improved.  It was also observed that the best location to carry out the attack is above the second vehicle and as attacker moves down the string, impact of attack on destabilization of platoon reduces.  

Tithi et al. \cite{TWG17} investigate whether a leader vehicle with monocular camera can leverage relative distance between shadows of two vehicles over time to verify position claims made by the vehicles. As a compromised vehicle can destabilize the entire platoon by sending false data, correct knowledge of position and velocity (PV) information of each vehicle can render the attack ineffective. Thus, a trusted leader in non-line of sight scenario examines the PV data received from vehicles and compare it against data obtained from shadow of vehicles to identify the attacker vehicle. However, the paper did not consider different weather condition, nightime, and moving vehicle platoon in their analysis. 

Gao et al. \cite{GRZ16}  proposed an effective cooperative message authentication protocol for preventing vehicle platoons from false data injection attacks. They also developed a method for optimizing system parameters which could enhance efficiency and security. 

Dunn et al. \cite{DMSGSL17} determine the conditions that an attacker can exploit to violate string stability of heterogeneous platoon of ACC or CACC vehicles. They also proved that after attack the platoon will continually deviate from the desired state even in the absence of additional input to the system. The attack is demonstrated in mixed-traffic scenario that have combination of automated and non-automated vehicles on the same lane. An active attacker having control of a vehicle coordinate with passive attackers in the platton (whose control laws it can modify remotely) to cause maximum traffic flow instability. 

Sajjad et al. \cite{SSG17} applied methods from game theory to the problem of vehicular platooning and illustrated the behavior of an attacker employing an optimal strategy.  An infinite time horizon, linear quadratic zeros-sum game was formulated whose saddle point solution was found using methods from optimal control. The various solutions to this problem demonstrated that it was hard to achieve a setting that sees one  attacker vehicle colliding with other cars or causing large oscillations in traffic flow while still fairing well in the game itself. Thus, it was usually not in the best interest of the attacker to disrupt regular operation. 
 




\section{Preliminaries and Problem Description} 
\label{sec:prelim}
\subsection{Car-Following Model of Truck Platoon}


\subsection{Attack Model}

We consider an adversary that can control $m$ out of $n$ trucks (where, $m \leq n/2$) of a Driver Assistive Truck Platoon (DATP). The trucks uses radar and vehicle-to-vehicle (V2V) communication to drive longitudinally in a close-headway formation at highway speeds and to maintain a safe gap with the preceding vehicle. By remotely exploiting vulnerability in $m$ truck's communication software, an attacker can hijack V2V's message transmission unit. Subsequently, by injecting malicious messages in the communication network of $(n-m)$ trucks, an adversary intends to impact the string stability of the platoon.  We assume that the leader of the platoon is not attacked and we do not know the $m$ attacked vehicles. Thus, the first goal is to detect the attacked trucks. The next goal is to identify the minimum number of trucks $m$, an adversary should attack to destabilize the platoon. 

In the second attack model, we consider multiple vehicles communicating via V2V in a multilane highway. The vehicles broadcast Basic Safety Messages (BSM) to its neighboring vehicles. An adversary in control of one of the V2V equipped car can broadcast corrupted BSM with the intention of influencing the direction of motion of the vehicles. Thus, the goal is to design control law that can guarantee resilient coordinated motion of vehicles \cite{SSPPK17}.       

%\subsection{Problem Description}





% \begin{itemize}
%     \item How does authentication work in VPKI? 
%     \item If a malicious car injects messages, how are the data validated or is any message accepted? 
%     \item Attack on sensor. Attack on network
%     \item How many vehicles are required for consensus algorithm to work correctly? 
%     \item Ramp condition for trucks not considered
% \end{itemize}



% \section{Methodology} \label{sec:method}
% \input{secs/Methodology}

%\section{Experimental Results} \label{sec:demo}
%\input{secs/Results}

%\vspace{-0.1cm}

%\section{Conclusion} \label{sec:conclude}
%\input{secs/6-conclusion}



%\appendices
%\section{Proof of the First Zonklar Equation}


% \section*{Acknowledgment}

% The work presented in this paper is supported by National Science Foundation, grant \#
% \ifCLASSOPTIONcaptionsoff
%   \newpage
% \fi

\bibliographystyle{ieeetr}
\bibliography{cps}


% \section{Notes}\label{sec:notes}
% \input{secs/7-notes}

%\section{Summary of Other Papers}
%\input{secs/2-priorart}
\end{document}

% Input: Observation y' in Algorithm 1 changed from y


