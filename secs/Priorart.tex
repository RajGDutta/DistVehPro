% Why they do it?  How they do? What they found? What it means? % what are the main differences? 
Dadras et al. \cite{DGRS15} study behavior of a vehicle platoon in an adversarial environment. They consider insider attack where an adversary has control of a vehicle of the platoon and whose derivative controller gain it can modify to destabilize or take control of the entire platoon. In the considered platoon, predecessor and follower vehicles distance and velocity information are used by a vehicle to compute its velocity and acceleration. The attack considered in the paper is feasible at every position of the platoon. 
%\begin{itemize}
% \item Questions and observations made in the paper
% \item Under adversarial environment stability and string stability are not same
% \item Controllability of platoon by an attacker vehicle at different position: a platoon is controllable by the attacker controlled vehicle only if it has full ranked controllability matrix and the uncontrollable states can be computed by performing similarity comparison.  
%\item The derivative gain of the controller should be within a range of values to cause instability of string
%\item Find the necessary and maximum value of attacker gain to cause instability of platoon ?
%\item Find the control input fequency that will destabilize the platoon?
%\item The platoon can be made asymptotically and BIBO unstable by making real part of single eigenvalue of matrix A > 0. 
%\item The platoon can be made to collapse at one target vehicle by attacking multiple vehicle of the platoon
%\end{itemize}
