% Why they do it?  How they do? What they found? What it means? % what are the main differences? 
Dadras et al. \cite{DGRS15} study behavior of a vehicle platoon in an adversarial environment. They consider insider attack where an adversary has control of a vehicle of the platoon and whose derivative controller gain it can modify to destabilize or take control of the entire platoon. In the considered platoon, predecessor and follower vehicles distance and velocity information are used by a vehicle to compute its velocity and acceleration. The attack considered in the paper is feasible at every position of the platoon. 
%\begin{itemize}
% \item Questions and observations made in the paper
% \item Under adversarial environment stability and string stability are not same
% \item Controllability of platoon by an attacker vehicle at different position: a platoon is controllable by the attacker controlled vehicle only if it has full ranked controllability matrix and the uncontrollable states can be computed by performing similarity comparison.  
%\item The derivative gain of the controller should be within a range of values to cause instability of string
%\item Find the necessary and maximum value of attacker gain to cause instability of platoon ?
%\item Find the control input fequency that will destabilize the platoon?
%\item The platoon can be made asymptotically and BIBO unstable by making real part of single eigenvalue of matrix A > 0. 
%\item The platoon can be made to collapse at one target vehicle by attacking multiple vehicle of the platoon
%\end{itemize}

Sajjad et al. \cite{SSDSG15} have designed an adversarial aware sliding mode control scheme that uses only local sensor data and a decentralized attack detector to reduce the number and severity of collision from an attacker controlled vehicle of a platoon. Their solution makes the assumption that the bidirectional platoon is homogeneous with all cars sharing thesame control law. They consider a single attacker that modifies the control law of a vehicle to induce an oscillatory behavior in the platoon. While preserving safety of the platoon, the method led to compromise of string stability.
%  They start the attack when platoon is in steady state. Size of the platoon is known and the controller gains are tuned according to it
% IN CACC absolute relative distance is measured by radar and acceleration information of the preceding vehicle is received through the wireless channel 

Fanid et al. \cite{FDZZ17} study the impact of channel and mobile jamming attack on the string stability of interconnected vehicles (homogeneous) equipped with  cooperative adaptive cruise control (CACC) system. The attacker (a drone) in their analysis jams the wireless channel among the vehicles to prevent the receivers from decoding the messages and thus cause destabilization of the platoon. They consider a two-ray path loss model to demonstrate that signal's power attenuation during the attack degrades performance of the CACC system and destabilizes the platoon. They found that by increasing headway time and using a memory block in vehicles CACC system, stability could be improved.  It was also observed that the best location to carry out the attack is above the second vehicle and as attacker moves down the string, impact of attack on destabilization of platoon reduces.  

Tithi et al. \cite{TWG17} investigate whether a leader vehicle with monocular camera can leverage relative distance between shadows of two vehicles over time to verify position claims made by the vehicles. As a compromised vehicle can destabilize the entire platoon by sending false data, correct knowledge of position and velocity (PV) information of each vehicle can render the attack ineffective. Thus, a trusted leader in non-line of sight scenario examines the PV data received from vehicles and compare it against data obtained from shadow of vehicles to identify the attacker vehicle. However, the paper did not consider different weather condition, nightime, and moving vehicle platoon in their analysis. 

Gao et al. \cite{GRZ16}  proposed an effective cooperative message authentication protocol for preventing vehicle platoons from false data injection attacks. They also developed a method for optimizing system parameters which could enhance efficiency and security. 

Dunn et al. \cite{DMSGSL17} determine the conditions that an attacker can exploit to violate string stability of heterogeneous platoon of ACC or CACC vehicles. They also proved that after attack the platoon will continually deviate from the desired state even in the absence of additional input to the system. The attack is demonstrated in mixed-traffic scenario that have combination of automated and non-automated vehicles on the same lane. An active attacker having control of a vehicle coordinate with passive attackers in the platton (whose control laws it can modify remotely) to cause maximum traffic flow instability. 

Sajjad et al. \cite{SSG17} applied methods from game theory to the problem of vehicular platooning and illustrated the behavior of an attacker employing an optimal strategy.  An infinite time horizon, linear quadratic zeros-sum game was formulated whose saddle point solution was found using methods from optimal control. The various solutions to this problem demonstrated that it was hard to achieve a setting that sees one  attacker vehicle colliding with other cars or causing large oscillations in traffic flow while still fairing well in the game itself. Thus, it was usually not in the best interest of the attacker to disrupt regular operation. 

Heijden et al. \cite{HLK17} developed a framework within which impact of various attacks and resilience of vehicle controllers could be analyzed. They performed their experiments on Plexe simulator that includes a constant spacing controller, a controller with graceful degradation, and a consensus controller. 

Xu et al. \cite{XLWZH17} proposed a dynamic ternary join-exit-tree (TJET) based dynamic key management scheme for secure communication in a vehicle platoon. The method efficiently updates keys of vehicles joining and leaving a platoon as well as of platoons merging and splitting. This mechanism prevents an attacker controlled vehicle from being part of the platoon i.e. prevents from active outside attack by implementing authetication, data integrity, forward security, backward security, and resistance to key control. However, they cannot prevent against insider attack i.e. a vehicle which is part of the platoon and is taken control by an adversary.  

 Kafash et al. \cite{KGMCR17} proposes an approach for limiting the capabilities of an attacker by imposing artificial bounds on the control inputs that drive the system. Whether the attack was caused by manipulation of the control inputs or sensors, these actuator bounds could restrict the system from reaching unsafe states. They derive methods based on convex optimization to quantify the reachable states (good states) given known actuator bounds and also develop methods to design new bounds to avoid the reachable set from entering a set of unsafe or dangerous states. 

Amoozadeh et al. \cite{ARCGZRL15} consider a two-level controller model and describe a variety of attacks, including application and network layer attacks, as well as other issues such as sensor tampering and privacy issues. They present a result of a message falsification attack, which in their setting is a vehicle external to the platoon that falsifies all beacons; the result of the attack is platoon instability (i.e., the platoon does not converge back to a stable state). They also explain that such an attack is most effective when acceleration changes occur. Similar results are presented for a jamming attack – here, the authors’ controller downgrades to ACC automatically. In our work, we aim to show that such a downgrade is not necessarily sufficient, depending on how system parameters are chosen.

DeBruhl et al. \cite{DWST15} used a two component controller, consisting of a PD feedback controller (radar-based) to keep distance and a feedforward controller (communication-based), whose outputs are added together to form the desired acceleration that is provided to the vehicle drive-train. This controller has been shown to be string-stable in real networked platoons, and they examine its behavior under various attack strategies, including a collision induction attack. The authors also develops an error calculation and detection algorithm, which essentially estimates the expected behavior of the vehicle in front and switches to ACC if this vehicle appears to behave differently than it claims to.

Gerdes et al. \cite{GWH13} examined adversarial behavior within an ACC-based platoon, where the attacker systematically disrupts the distributed control algorithm using another algorithm that manipulates control input of their own vehicle to achieve their goal. They show that for a PD controller, the attacker can cause the platoon to be asymptotically unstable, i.e., destabilize and eventually dissolve the platoon.

Liu et al. \cite{LMWZ17} proposes a safety-security co-design process to derive functional security requirements for a safe automated homogeneous vehicle platoon system. They have developed a method that switches from CACC to ACC mode when a insider attack is detected. In their adversary model, an attacker can control one or more vehicles, but it cannot corrupt the sensors of the vehicles in the platoon. By relying on data of the sensors and comparing it against data from vehicle-to-vehicle network, they were able to detect attacks that intend to violate the safe distance between vehicles.
 

